\documentclass[10pt]{beamer}
\usetheme{Warsaw}

\setbeamertemplate{footline}{}
\setbeamersize{text margin left=3.5mm,text margin right=3.5mm} 

\usepackage{hyperref}
\hypersetup{
    colorlinks=true,
    linkcolor=blue,
    filecolor=magenta,      
    urlcolor=blue,
}
 
\urlstyle{same}

\usepackage{datetime}
\newdateformat{specialdate}{\twodigit{\THEDAY}.\twodigit{\THEMONTH}.\THEYEAR}

\usepackage[utf8]{inputenc}

\usepackage{float}
\usepackage{graphicx}
\graphicspath{ {./images/} }

\usepackage{verbatim}

\usepackage{caption}
\captionsetup[figure]{name=Fig.}


\title{Easy AlphaFold  \\
Package for fast and convenient summary of the AlphaFold data}
\author{Maciej Sikora, Małgorzata Sudoł, Maria Bochenek, Kamil Pawlicki}



\begin{document}
\frame{\titlepage}


\begin{frame}
\frametitle{Project main goal}
Easy-AlphaFold-Report is a package with the aim of processing data between databases easier and faster for the user.\newline

Users can provide input in multiple formats - Pfam families, clans, Uniprot or PDB. Additionally, for even easier access for less tech-savvy users, the main script enables the use of a convenient GUI for a more click-oriented experience.\newline
\end{frame}


\begin{frame}
\frametitle{Project main goal}
As a result, users can expect an interactive and easy to read report with a summary of a provided dataset with a focus on AlphaFold database features. To help with the analysis, basic statistics are also calculated for a quick overview.\newline
\end{frame}

\begin{frame}
\frametitle{Project main goal}
Instead of clicking through multiple sites, and finding connections between databases, this package will do the work for you and if you still want to verify the results, or grab more data links on the report can send you to the respective websites.\newline

Optionally during the summary process, the user can also choose to keep the data from AlphaFold in the .cif files (cif is a more descriptive alternative for PDB files).
\end{frame}


\begin{frame}
\frametitle{Project features}
\begin{itemize}
\item Flexible input providing (direct input, files).
\item Convenient usage via GUI.
\item Various data type recognition and parsing (Pfam families, clans, PDB, Uniprot).
\item Automatic database updates.
\item ID verification.
\item Data collection from multiple databases (PDB, PFAM, Uniprot, AlphaFold).
\item Optional data download from AlphaFold (.cif files)
\item Presenting results and statistics in the intuitive and interactive html.
\end{itemize}
\end{frame}



\begin{frame}
\frametitle{Usage}

For the basic version with GUI user needs to simply run the main python script:
\begin{itemize}
\item python Easy\_AlphaFold.py
\end{itemize}

This should open an interactive window ready to use. \newline\newline

For programmatic access, terminal input is also available and uses the same script as a base - but with manual flags.\newline
Description of the flags can be found in the Github README.md file.\newline\newline

Input\newline
In the interactive version, the user can provide ID manually via a dedicated box or use a file selector to provide a path to the file. The file should contain 1 ID per line.
\end{frame}


\begin{frame}
\frametitle{}
\end{frame}


\end{document}
